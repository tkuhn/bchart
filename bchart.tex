\documentclass{article}

\usepackage{bchart}
\usepackage{graphicx}

\title{\texttt{bchart}: Simple Bar Charts in \LaTeX}
\author{Tobias Kuhn}
\date{26 August 2011}

\begin{document}

\maketitle

\begin{abstract}
\texttt{bchart} is a {\LaTeX} package for drawing simple bar charts with horizontal bars on a numerical x-axis. It is based on the TikZ drawing package. The focus of this package is on simplicity and aesthetics.
\end{abstract}


\section{Introduction}

\texttt{bchart} is a package for drawing bar charts within {\LaTeX} documents. The only type of charts supported by this package are bar charts with horizontal bars on a numerical x-axis. The creation of such charts is very simple, as will be shown below. To use the package, you have to make sure that {\LaTeX} is able find the file \texttt{bchart.sty}, e.g. by placing a copy of it into the directory of the \texttt{tex}-file that is using the package. In order to load the package, you have to place the following command at the beginning of your {\LaTeX} source file:
\begin{quote}\small
\begin{verbatim}
\usepackage{bchart}
\end{verbatim}
\end{quote}


\section{Charts}

Charts are created with the \texttt{bchart}-environment. Within this environment, you can put one or more bars by using the \texttt{bcbar}-command. A very simple chart can look as follows (code and resulting chart):
\begin{quote}\small
\begin{verbatim}
\begin{bchart}[max=50]
  \bcbar{45}
  \bcbar{26}
  \bcbar{31}
\end{bchart}
\end{verbatim}
\end{quote}
\begin{quote}
\begin{bchart}[max=50]
  \bcbar{45}
  \bcbar{26}
  \bcbar{31}
\end{bchart}
\end{quote}
The option \texttt{max} defines the maximum value on the x-axis (the default is 100). The number given to the \texttt{bcbar}-command defines the value of the respective bar, visualized by the width of the bar and printed as a number to the right of it. The minimum value on the x-axis defaults to 0, but this can be changed by the use of the option \texttt{min}. The option \texttt{step} can be used to show marks at a regular interval:
\begin{quote}\small
\begin{verbatim}
\begin{bchart}[min=2.5,step=0.25,max=3.75]
  \bcbar{2.6}
  \bcbar{3.7}
  \bcbar{3.1}
\end{bchart}
\end{verbatim}
\end{quote}
\begin{quote}
\begin{bchart}[min=2.5,step=0.25,max=3.75]
  \bcbar{2.6}
  \bcbar{3.7}
  \bcbar{3.1}
\end{bchart}
\end{quote}
For marks at irregular intervals, the \texttt{steps}-option can be used instead:
\begin{quote}\small
\begin{verbatim}
\begin{bchart}[min=1,max=18,steps={1,3,7,15}]
  \bcbar{13}
  \bcbar{4}
  \bcbar{7}
\end{bchart}
\end{verbatim}
\end{quote}
\begin{quote}
\begin{bchart}[min=1,max=18,steps={1,3,7,15}]
  \bcbar{13}
  \bcbar{4}
  \bcbar{7}
\end{bchart}
\end{quote}
To show no marks at all, the flag \texttt{plain} is used:
\begin{quote}\small
\begin{verbatim}
\begin{bchart}[max=8,plain]
  \bcbar{6.2}
  \bcbar{1.8}
\end{bchart}
\end{verbatim}
\end{quote}
\begin{quote}
\begin{bchart}[max=8,plain]
  \bcbar{6.2}
  \bcbar{1.8}
\end{bchart}
\end{quote}


\section{Bars}

The bars of a bar chart can be modified in several ways. The \texttt{text}-option prints text within the inside of the bar:
\begin{quote}\small
\begin{verbatim}
\begin{bchart}[step=2,max=8]
  \bcbar[text=Year 1]{6}
  \bcbar[text=Year 2]{3}
\end{bchart}
\end{verbatim}
\end{quote}
\begin{quote}
\begin{bchart}[step=2,max=8]
  \bcbar[text=Year 1]{6}
  \bcbar[text=Year 2]{3}
\end{bchart}
\end{quote}
The background color of the bar can be changed with the \texttt{color}-option:
\begin{quote}\small
\begin{verbatim}
\begin{bchart}[max=8]
  \bcbar[text=A,color=yellow]{6}
  \bcbar[text=B,color=red!50]{3}
  \bcbar[text=C,color=green!60!blue]{4}
\end{bchart}
\end{verbatim}
\end{quote}
\begin{quote}
\begin{bchart}[max=8]
  \bcbar[text=A,color=yellow]{6}
  \bcbar[text=B,color=red!50]{3}
  \bcbar[text=C,color=green!60!blue]{4}
\end{bchart}
\end{quote}
In addition to predefined colors like \texttt{yellow} and \texttt{red}, new colors can be defined and existing ones can be combined. \texttt{red!50}, for example, means a 50\% saturated red, whereas \texttt{green!60!blue} stands for a color obtained by blending green and blue at a ratio of 60:40. See the TikZ manual for more information on how to manipulate colors. The default color for bars is \texttt{blue:20}. The \texttt{plain}-flag can be used to prevent the value of the bar to be displayed:
\begin{quote}\small
\begin{verbatim}
\begin{bchart}[step=1,max=8]
  \bcbar[plain]{6}
  \bcbar[plain]{3}
\end{bchart}
\end{verbatim}
\end{quote}
\begin{quote}
\begin{bchart}[step=1,max=8]
  \bcbar[plain]{6}
  \bcbar[plain]{3}
\end{bchart}
\end{quote}


\section{Skips}

Small, medium and big skips:
\begin{quote}\small
\begin{verbatim}
\begin{bchart}[step=2,max=10]
  \bcbar{3.4}
  \smallskip
  \bcbar{5.6}
  \medskip
  \bcbar{7.2}
  \bigskip
  \bcbar{9.9}
\end{bchart}
\end{verbatim}
\end{quote}
\begin{quote}
\begin{bchart}[step=2,max=10]
  \bcbar{3.4}
  \smallskip
  \bcbar{5.6}
  \medskip
  \bcbar{7.2}
  \bigskip
  \bcbar{9.9}
\end{bchart}
\end{quote}
Skips with \texttt{bcskip}:
\begin{quote}\small
\begin{verbatim}
\begin{bchart}[step=10,max=100]
  \bcbar{83}
  \bcskip{3pt}
  \bcbar{25}
  \bcskip{15mm}
  \bcbar{69}
\end{bchart}
\end{verbatim}
\end{quote}
\begin{quote}
\begin{bchart}[step=10,max=100]
  \bcbar{83}
  \bcskip{3pt}
  \bcbar{25}
  \bcskip{15mm}
  \bcbar{69}
\end{bchart}
\end{quote}


\section{Labels}

Label for x axis:
\begin{quote}\small
\begin{verbatim}
\begin{bchart}[step=200,max=1000]
  \bcbar[text=Company A]{346}
  \bcbar[text=Company B]{873}
  \bcxlabel{number of employees}
\end{bchart}
\end{verbatim}
\end{quote}
\begin{quote}
\begin{bchart}[step=200,max=1000]
  \bcbar[text=Company A]{346}
  \bcbar[text=Company B]{873}
  \bcxlabel{number of employees}
\end{bchart}
\end{quote}

Labels for bars and skips:
\begin{quote}\small
\begin{verbatim}
\begin{bchart}[step=2,max=16]
  \bcbar[label=1st bar label]{8.5}
  \bigskip[label=skip label]
  \bcbar[label=2nd bar label]{4.5}
\end{bchart}
\end{verbatim}
\end{quote}
\begin{quote}
\begin{bchart}[step=2,max=16]
  \bcbar[label=1st bar label]{8.5}
  \bigskip[label=skip label]
  \bcbar[label=2nd bar label]{4.5}
\end{bchart}
\end{quote}

Free labels:
\begin{quote}\small
\begin{verbatim}
\begin{bchart}[step=2,max=16]
  \bcbar{8.5}
  \bclabel[label=skip label]
  \bcbar{4.5}
\end{bchart}
\end{verbatim}
\end{quote}
\begin{quote}
\begin{bchart}[step=2,max=16]
  \bcbar{8.5}
  \bclabel{free label}
  \bcbar{4.5}
\end{bchart}
\end{quote}


\section{Units}

Units:
\begin{quote}\small
\begin{verbatim}
\begin{bchart}[min=50,max=100,step=10,unit=\%]
  \bcbar{62.3}
  \bcbar{81.6}
\end{bchart}
\end{verbatim}
\end{quote}
\begin{quote}
\begin{bchart}[min=50,max=100,step=10,unit=\%]
  \bcbar{62.3}
  \bcbar{81.6}
\end{bchart}
\end{quote}


\section{Width and Scaling}

Width:
\begin{quote}\small
\begin{verbatim}
\begin{bchart}[max=10,step=2,width=4cm]
  \bcbar{7.5}
  \bcbar{3.2}
\end{bchart}
\end{verbatim}
\end{quote}
\begin{quote}
\begin{bchart}[max=10,step=2,width=4cm]
  \bcbar{7.5}
  \bcbar{3.2}
\end{bchart}
\end{quote}

Scale:
\begin{quote}\small
\begin{verbatim}
\begin{bchart}[max=10,step=2,scale=0.7]
  \bcbar{7.5}
  \bcbar{3.2}
\end{bchart}
\end{verbatim}
\end{quote}
\begin{quote}
\begin{bchart}[max=10,step=2,scale=0.7]
  \bcbar{7.5}
  \bcbar{3.2}
\end{bchart}
\end{quote}

Scale:
\begin{quote}\small
\begin{verbatim}
\scalebox{0.7}{
\begin{bchart}[max=10,step=2]
  \bcbar{7.5}
  \bcbar{3.2}
\end{bchart}}
\end{verbatim}
\end{quote}
\begin{quote}
\scalebox{0.7}{
\begin{bchart}[max=10,step=2]
  \bcbar{7.5}
  \bcbar{3.2}
\end{bchart}}
\end{quote}

Combined:
\begin{quote}\small
\begin{verbatim}
\scalebox{0.7}{
\begin{bchart}[max=10,step=2,width=4cm,scale=0.7]
  \bcbar{7.5}
  \bcbar{3.2}
\end{bchart}}
\end{verbatim}
\end{quote}
\begin{quote}
\scalebox{0.7}{
\begin{bchart}[max=10,step=2,width=4cm,scale=0.7]
  \bcbar{7.5}
  \bcbar{3.2}
\end{bchart}}
\end{quote}


\section{Known Issues}

\begin{quote}\small
\begin{verbatim}
\begin{bchart}[step=0.2,max=1]
  \bcbar{0.76}
\end{bchart}
\end{verbatim}
\end{quote}
\begin{quote}
\begin{bchart}[step=0.2,max=1]
  \bcbar{0.76}
\end{bchart}
\end{quote}

\begin{quote}\small
\begin{verbatim}
\begin{bchart}[steps={0.2,0.4,0.6,0.8,1},max=1]
  \bcbar{0.76}
\end{bchart}
\end{verbatim}
\end{quote}
\begin{quote}
\begin{bchart}[steps={0.2,0.4,0.6,0.8,1},max=1]
  \bcbar{0.76}
\end{bchart}
\end{quote}

\end{document}

